\documentclass[11pt]{article}

% fonts
\usepackage[utf8]{inputenc}
\usepackage[T1]{fontenc}
\usepackage[sc]{mathpazo}

% spacing
\usepackage[margin=1in]{geometry}
\setlength{\parskip}{1ex}
\usepackage{multicol}
\usepackage{setspace}
\usepackage{tabularx}
\onehalfspacing

% orphans and widows
\clubpenalty=10000
\widowpenalty=10000

%------------------------------------------------------------------------------%
\begin{document}

%% Insert the name of your project, the name of your team, and the name and email of each student.

\begin{center}
\bfseries\huge
Parallel Approaches to the N-Bodies Problem
\end{center}

\begin{multicols}{4}
\centering

Ben Bole \\
{\footnotesize bolebj@dukes.jmu.edu}

John Latino \\
{\footnotesize latinojr@dukes.jmu.edu}

Richard Bimmer \\
{\footnotesize bimmerra@dukes.jmu.edu}

Kevin Kelly \\
{\footnotesize kelly4kc@dukes.jmu.edu}

\end{multicols}

%------------------------------------------------------------------------------%
\begin{center}
\section*{Introduction}
\end{center}

%% This section is to introduce what exactly the n-bodies problem is, why it is an important and relevant problem, and its relevancy to HPC.

The N-bodies problem is a classical physics problem in which the goal is to predict the motion of N celestial bodies, given starting conditions. 
%%Make sure the starting conditions are correct 
These starting conditions include the (x, y, z) coordinates, starting velocities, and masses of each of the N celestial bodies. When N $\le$ 2, this problem is trivial, however, when N $\ge$ 3, prediction of the motions of bodies becomes much more computationally intensive. There are multiple different methods for efficiently paralellizing the computation of gravitational forces between celestial bodies.
\newline \newline
In this paper, we investigate three techniques for approaching the N-bodies problem: the Barnes-Hut Simulation, the Parker-Sochacki Method, and the Particle-Particle-Particle Mesh (P3M) Method. We will be modifying already existing code bases of each respective approach, so that the input and output of each respective program is in the same form. We will be analysing run-time and scalability of each approach in order to determine which approach would be the best to iterate upon.
%------------------------------------------------------------------------------%
\begin{center}
\section*{The Barnes-Hut Approach}
\end{center}
%This section is to describe the Barnes-Hut Method and to (in the future) describe performance of the Barnes-Hut Method, serially, parallel, and (possibly) using CUDA.
The Barnes-Hut Simulation (BHS) seeks treat the 3-dimensional space as a tree, specifically an octree. In an octree, each internal node of an octree stores exactly eight children. Usually used in the realm of 3-dimensional graphics/simulations, the eight children of each internal node represents the octants of the original 3D space of the parent node. In BHS, each internal node stores a reference to its center of mass, as well as its eight children - internal nodes do not care about individual bodies. 
\newline \newline 
On each time step, the following steps are performed. Space is recursively divided up until each leaf node either contains one or fewer bodies within its bounds. Net force on each body is calculated. This is done by traversing the tree, starting at the starting node. It is then determined if the center of mass for that internal node is significantly close to the body. If the center of mass is not significantly close, then calculations are performed using the center of mass of the internal nodes as a reference. If the center of mass is significantly close, that node's children are traversed; if the node is a leaf node, calculations on the bodies themselves are performed. This process is performed until the specified number of time steps are performed.
\newline \newline
Because of the intensive computation required in dividing up space on each time step, BHS is normally used to simulate an extremely large space in which bodies are very far from each other. This makes it so that the algorithm can frequently "skip" calculations on individual bodies, in noting the significance of the distance between the body and internal nodes' "center of mass".
\newline \newline
TODO: ANALYSIS AND SCALING OF BHS
%------------------------------------------------------------------------------%
\begin{center}
\section*{The Parker-Sochacki Approach}
\end{center}
%This section is to describe the Parker-Sochacki method and explain the math behind it. 
The Parker Sochacki Method (PSM) seeks to manage the interactions between celestial bodies by manipulating ordinary differential equations (ODEs). Parker and Sochacki proved several years ago that it is possible to manipulate any arbitrary ODE into a system of polynomial ODEs. Having a system of polynomial ODEs then allows us to shift those ODEs into a metric space, allowing for much easier and faster mathematical manipulation. 
\newline \newline
Since we constructed our metric space using polynomial ODEs - all of which were designed from our initial ODE - we know that our metric space is a Cauchy sequence, making it a complete metric space. Since our constructed metric space is complete, by the Banach-Caccioppoli fixed point theorem, we know that each of our ODEs, once plugged into an iterated function $f$, will converge onto a fixed point, $x$ - different for each ODE. The fixed point $x$ is the solution to our originial ODE. 
\newline \newline 
This is an extremely convenient method for approaching the N-bodies problem, since the most computationally intensive action is the performance of the iterated function on the metric space. Since we only require the initial ODE, this process of iteration is highly parallelizable.
\newline \newline
TODO: ANALYSIS AND SCALING OF PSM
%------------------------------------------------------------------------------%
\begin{center}
\section*{The P3M Approach}
\end{center}
The Particle-Particle-Particle Mesh (P3M) Method is a unique approach because it was originally designed to address the n-bodies problem not at a celestial scale, but at a molecular one. The P3M approach is built on top of a now obsolete method called Particle Mesh.
\newline \newline
Particle Mesh took in the positions of all bodies given to it and then interpolated those positions onto a grid. Once all positions were populated by the bodies, the forces acting on each point in the grid would be calculated using Poisson's equation. This method could, within a realm of error, accurately simulate N-bodies which did not share the same grid position, however the same could not be said for particles which did share a grid position.
\newline \newline
In order to alleviate this issue, P3M requires calculation of the forces of each individual body which shares a spot on the grid. Like BHS, P3M performs faster when bodies are farther apart from each other, due to the individual calculations which must be performed on each body sharing a grid location.
\newline \newline
TODO: ANALYSIS AND SCALING OF P3M

%------------------------------------------------------------------------------%
\begin{center}
\section*{Comparison}
\end{center}
TODO: COMPARISONS OF EACH APPROACH
%------------------------------------------------------------------------------%
\begin{center}
\section*{Progress (Mid-Project Only)}
\end{center}

\end{document}