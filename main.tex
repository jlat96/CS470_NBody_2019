\documentclass[11pt]{article}

% fonts
\usepackage[utf8]{inputenc}
\usepackage[T1]{fontenc}
\usepackage[sc]{mathpazo}

% spacing
\usepackage[margin=1in]{geometry}
\setlength{\parskip}{1ex}
\usepackage{multicol}
\usepackage{setspace}
\usepackage{tabularx}
\onehalfspacing

% orphans and widows
\clubpenalty=10000
\widowpenalty=10000

%------------------------------------------------------------------------------%
\begin{document}

%% Insert the name of your project, the name of your team, and the name and email of each student.

\begin{center}
\bfseries\huge
Parallel Approaches to the N-Bodies Problem
\end{center}

\begin{multicols}{4}
\centering

Ben Bole \\
{\footnotesize bolebj@dukes.jmu.edu}

John Latino \\
{\footnotesize latinojr@dukes.jmu.edu}

Richard Bimmer \\
{\footnotesize bimmerra@dukes.jmu.edu}

Kevin Kelly \\
{\footnotesize kelly4kc@dukes.jmu.edu}

\end{multicols}

%------------------------------------------------------------------------------%
\begin{center}
\section*{Introduction}
\end{center}

%% This section is to introduce what exactly the n-bodies problem is, why it is an important and relevant problem, and its relevancy to HPC.

The N-bodies problem is a classical physics problem in which the goal is to predict the motion of N celestial bodies, given starting conditions. 
%%Make sure the starting conditions are correct 
These starting conditions include the (x, y, z) coordinates, starting velocities, and masses of each of the N celestial bodies. When N $\le$ 2, this problem is trivial, however, when N $\ge$ 3, prediction of the motions of bodies becomes much more computationally intensive. There are multiple different methods for efficiently paralellizing the computation of gravitational forces between celestial bodies.
\newline \newline
In this paper, we investigate three techniques for approaching the N-bodies problem: the Barnes-Hut Simulation, the Parker-Sochacki Method, and the Particle-Particle-Particle Mesh (P3M) Method. We will be modifying already existing code bases of each respective approach, so that the input and output of each respective program is the same. We will be analysing run-time and scalability of each approach in order to determine which approach would be the best to iterate upon.
%------------------------------------------------------------------------------%
\begin{center}
\section*{The Barnes-Hut Approach}
\end{center}
%This section is to describe the Barnes-Hut Method and to (in the future) describe performance of the Barnes-Hut Method, serially, parallel, and (possibly) using CUDA.
The Barnes-Hut Simulation (BHS) seeks treat the 3-dimensional space as a tree, specifically an octree. In an octree, each internal node of an octree stores exactly eight children. Usually used in the realm of 3-dimensional graphics/simulations, the eight children of each internal node represents the octants of the original 3D space of the parent node. In BHS, each internal node stores a reference to its center of mass, as well as its eight children - internal nodes do not care about individual bodies. 
\newline \newline 
On each time step, the following steps are performed. Space is recursively divided up until each leaf node either contains one or fewer bodies within its bounds. Net force on each body is calculated. This is done by traversing the tree, starting at the starting node. It is then determined if the center of mass for that internal node is significantly close to the body. If the center of mass is not significantly close, then calculations are performed using the center of mass of the internal nodes as a reference. If the center of mass is significantly close, that node's children are traversed; if the node is a leaf node, calculations on the bodies themselves are performed. This process is performed until the specified number of time steps are performed.
\newline \newline
Because of the intensive computation required in dividing up space on each time step, BHS is normally used to simulate an extremely large space in which bodies are very far from each other. This makes it so that the algorithm can frequently "skip" calculations on individual bodies, in noting the significance of the distance between the body and internal nodes' "center of mass".
%------------------------------------------------------------------------------%
\begin{center}
\section*{The Parker-Sochacki Approach}
\end{center}
%This section is to describe the Parker-Sochacki method and explain the math behind it. 


%------------------------------------------------------------------------------%
\begin{center}
\section*{The P3M Approach}
\end{center}


%------------------------------------------------------------------------------%
\begin{center}
\section*{Analysis}
\end{center}
TODO
%------------------------------------------------------------------------------%
\begin{center}
\section*{Progress}
\end{center}

\end{document}